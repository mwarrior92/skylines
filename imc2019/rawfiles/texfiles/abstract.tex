\begin{abstract}

``Do you see what I see?'' The vastness of today's Internet creates an intuitive
but often overlooked phenomenon: not everyone is exposed to the same web
resources. Even across the set of objects embedded in a single web page, a pair of
clients with apparently similar network properties may be assigned to barely
overlapping sets of network resources to pull from. While the properties of
individual content distribution networks (CDNs) and the like are well explored,
there has been, until now, a lack of insight regarding the \emph{aggregate}
behavior of these many large networks co-existing.  

In this paper, we perform the first, deep analysis of cross-provider resource
allocation patterns and the resulting aggregate mapping of over 10,000 RIPE
Atlas clients around the world. To facilitate our research, we introduce common
network resource exposure (CNRE) - a measure of the degree to which a pair of
clients are exposed to the same network destinations as each other across a
large set of domains. We explore the implications of high and low CNRE scores,
and assess the relationship between CNRE and well established network properties (country,
ASN, and BGP prefix). 
    
By clustering clients that share high CNRE measures, we identify aggregate
    catchment patterns encompassing 302 popular domains. Our findings expose clients that
are poorly served by their current position in this aggregate mapping scheme and
show that a client's distance from the geographic center of their CNRE cluster is
directly proportional to the latency they experience across providers.

\end{abstract}


