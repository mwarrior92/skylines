\subsection{Problem Space and Related Work} \label{skyspace}

This projects aims to gain an understanding of which clients are directed to the same set of
resources across many distinct domains. Its most direct and immediate use case is influencing probe
selection in large scale Internet measurements. For researchers, likely unaware of the relatively
hidden allocation schemes of the wide array of CDN platforms and other large content distributors,
it is difficult to determine, a priori, the degree of similarity between clients. Knowledge of
whether there is a high probability that a pair of clients are being directed to altogether
different resources may be significant to their experiment design. This approach to experiment
design is in line with RIPE Atlas, one of the largest client based measurement platforms,
which maintains
an exhaustive set of tags on all of their clients in order to help researchers and network operators
filter and refine the set selected for their experiment \cite{ripe-atlas}. Further, more abstract
applications may include, but are not limited to, distributed denial of service mitigation
\cite{anycastvsddos} and CDN node deployment \cite{35590, Tariq}.

The most similar body of related work involves anycast CDN catchment analysis, which aims to
investigate the set of clients routed towards particular CDN points of presence (PoPs)
\cite{Calder2015, anycastvsddos, vdmscatchment}. My work differs significantly in scope: to my
knowledge, I am the first investigate what I refer to as \emph{aggregate catchments}, the joint
behavior of many anycast CDN catchments and unicast CDN targets, spread across many content
distribution platforms. Conversely, this related body work either focuses on individual platforms or
specific services \cite{Calder2015, anycastvsddos, vdmscatchment}. 

Several authors have attempted to discover the topology of large CDN platforms through large scale
measurement studies \cite{webcart, Calder2013, benson11}. While their findings are potentially of
use in this project, their goals and contributions run parallel to what I aim to accomplish. They
seek to identify the properties and locations of CDN resources; conversely, I seek to identify the
target pools (sets of clients) of overlapping CDN resource catchments \cite{webcart, Calder2013,
benson11}. Other work close to this space investigates the performance of a particular CDN
deployment scheme \cite{ecs15sigcomm}. 
