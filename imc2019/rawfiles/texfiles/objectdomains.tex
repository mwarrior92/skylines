\subsection{Why Web Object Domains}

In Section \ref{domcollect}, we described how the set of domains used in our
experiment were selected. The selected domains were extracted directly
from web object URLs observed across the set of web pages we checked. We note
here that these domains are often abstractions of more explicit hosting schemes.
For example, such domains may resolve to unique CNAMES CITE or directly resolve
to third party CDN addresses CITE. In contrast to our approach, we could have
converted each domain into some lower level representation (such as its CDN) and
in turn performed a CNRE-like measurement study using this representation
(\emph{i.e.}, ping from each client to each CDN in our set).

While this alternative approach may provide its own insights, we chose leave
each discovered domain ``as is'' for two reasons. First, although a given domain
may use CDN hosting, it is worth noting that modern CDN selection techniques are
complex and diverse. While some content providers may opt to utilize a
\emph{single} CDN for their purposes, it has also become common practice to
instead depend on CDN \emph{brokers} or \emph{multi} CDNs to dynamically (by
cost, performance, geography, \emph{etc.}) make use of a set of CDNs. In other
words, the ``less abstract'' representation of a given content provider is, in
many cases, far from comprehensive with regard to a diverse set of clients.
Second, even if a domain or content provider could adequately be reduced to a
lower level description as described, the performance of a given CDN ``in
general'' is not necessarily representative of the performance of one of its
customers. The reasons for this are plentiful, ranging from customer specific
policies and agreements (for example, the customer might purchase region
specific support) to caching algorithms and load balancing. 
