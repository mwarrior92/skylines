\section{Conclusion}

As a byproduct of the growing levels of complexity and abstraction in today's
Internet, it is often taken for granted what clients sharing some labels or
properties will have in common. In this paper, we measured the extent to which
clients accessing the same websites share web resources. We performed over 52
million comparisons across more than 9,000 clients, and based our experiment in
the context of real resolution sets associated with popular webpages. To
faciliate our analysis, we introduced common network resource exposure (CNRE),
which quantifies the degree of overlap between the sets of network resources
observed by a pair of clients. We contextualized CNRE with respect to network
performance and existing client labeling conventions: BGP prefix, ASN, and
country. Our findings formally demonstrate that use of these existing labeling
schemes, in alternative to CNRE, fail to adequately capture the same
information.

Through CNRE, we were able to cluster clients by similarity and analyze 
relationships between members of the same cluster. We further identified each
cluster's
effective geographic center, each individually representative of the hundreds
of domains and thousands of measurements from which they were derived. Finally,
we validated geographic centers by observing a directly proportional
relationship between a client's distance from its cluster's center and the
client's average latency toward providers. Together,
the techniques presented in this paper --- which we collectively refer to as the
``Skyline model'' --- offer a powerful paradigm through which we can better
understand network resource allocation and performance patterns. 
