\section{Experiment Overview} \label{oversky}

I aim to establish a baseline by which to compare client labeling systems relative to how well they
capture the similarity of clients with regards to their DNS query answers across a large set of
domains. Clients exposed to the same set of resources should appear to be similar and thus grouped
together by a labeling scheme. Likewise, clients exposed to different resources should appear
disimilar. With this in mind, I opt to use the Jaccard index as inspiration for a simple distance
measure, which I will refer to as closeness. Closeness is explicitly defined in Section
\ref{closeness}.
Essentially, clients with high closeness scores have a large number of network resources in common
(a high degree of CNRE). 

In Section \ref{skyfinds}, I calculate pairwise closeness scores for thousands of clients spread across the
globe. I then group results by a series of conventional naming systems (prefix, /24, AS number, and
country) for further analysis. The continuity of possible closeness values --- which span from 0 to
1 --- give us a mechanism by which we can gauge how well the various labeling schemes are aligned
with the degree overlap in clients' resource exposure. If a labeling scheme is well aligned, we
should be able to expect clients with the same label to be directed to most of the same
endpoints. 

Given that closeness is a distance measure, it may be possible to derive some manner of grouping
from the matrix of pairwise results. In Section \ref{propsky}, I propose applying clustering techniques to
the dataset to gain an understanding of the broader relationship between clients in terms of their
closness. This manner grouping clients will form the basis of what I refer to as the Skyline model.

\subsection{Closeness Definition} \label{closeness}

Consider two clients, $p$ and $q$,  who have each resolved $d$ domains, taking the first answer from
each query.  Suppose I were to compare the each client's remaining answer for the first domain,
$d_0$. If the clients were directed to the same resource, client $p$'s answer, $A_0p$, should match
client $q$'s answer, $A_0q$. If I were to continue performing such comparisons across all $d$
domains, I would arrive with a list of booleans, attributing matches result in true values and
mismatches to false.  I define the closeness of two clients as $t / d$, where $t$ is the number of
true values from the list of booleans. 

For now and through my preliminary experiements, I consider only answers within the
same /24 prefix to be \emph{matching}, as /24 is generally the largest routable prefix in IPv4.
However, it is worth noting that it is very possible for large datacenters to include multiple /24
blocks or smaller prefixes. I address this further in the Proposal portion of this document.


