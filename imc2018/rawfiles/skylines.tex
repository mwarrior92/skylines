\documentclass[10pt, sigconf]{acmart}

\let\oldbibitem\bibitem
\def\bibitem{\vfill\oldbibitem}

\usepackage{booktabs} % For formal tables
\usepackage{epsfig,endnotes,subcaption,url}
\usepackage{flushend}


% Copyright
%\setcopyright{none}
%\setcopyright{acmcopyright}
%\setcopyright{acmlicensed}
%\setcopyright{rightsretained}
%\setcopyright{usgov}
%\setcopyright{usgovmixed}
%\setcopyright{cagov}
%\setcopyright{cagovmixed}


\begin{document}
\title[Skylines]{Skylines: Pragmatic, Perspective-Based Internet Mapping}
%\titlenote{Produces the permission block, and copyright information}
%\subtitle{Extended Abstract}

\author{Marc Anthony Warrior}
\affiliation{%
  \institution{Northwestern University}
}
\email{warrior@u.northwestern.edu}

\author{Romain Fontugne}
\affiliation{%
  \institution{IIJ Innovation Institute}
}
\email{romain@iij.ad.jp}
\author{Randy Bush}
\affiliation{%
  \institution{IIJ Innovation Institute}
}
\email{randy@psg.com}

\renewcommand{\shortauthors}{M. Warrior et al.}

\begin{abstract}

    Existing mapping systems and measures used to describe Internet location
    fail to account for complexities introduced by DNS-based content
    allocation schemes. With DNS redirection, in particular, clients residing in
    the exact same network location may potentially be served by completely
    different content sources. Likewise, distant clients may be directed to the
    same servers. The existence of such scenarios challenges the validy of
    some common assumptions about network behavior: the relationship between a
    client's network location and its ``view'' of the Internet, independent of
    performance, is neither predictable nor consistent across clients. 

    In this paper, we introduce skylines, a distance measure that describes
    variation in Internet resource allocation between clients. TODO: finish list
    of contributions

\end{abstract}

%
% The code below should be generated by the tool at
% http://dl.acm.org/ccs.cfm
% Please copy and paste the code instead of the example below. 
%

\copyrightyear{2018} 
\acmYear{2018} 
\setcopyright{acmcopyright}
\acmConference[IMIC '18]{IMC '18: ACM Internet Measurement Conference 2018,
October 31--November 2, 2018}
\acmBooktitle{IMC '18: ACM Internet Measurement Conference 2018,
October 31--November 2, 2018}
\acmPrice{15.00}
\acmDOI{0.0/0.0}
\acmISBN{0-0-0-0-0/0/0}

\begin{CCSXML}
<ccs2012>
    <concept>
        <concept_id>#.#.#</concept_id>
        <concept_desc>Networks~Application layer protocols</concept_desc>
        <concept_significance>300</concept_significance>
    </concept>
    <concept>
        <concept_id>#.#.#</concept_id>
        <concept_desc>Networks~Naming and addressing</concept_desc>
        <concept_significance>100</concept_significance>
    </concept>
    <concept>
        <concept_id>#.#.#</concept_id>
        <concept_desc>Networks~Location based services</concept_desc>
        <concept_significance>100</concept_significance>
    </concept>
</ccs2012>
\end{CCSXML}

\ccsdesc[300]{Networks~Application layer protocols}
\ccsdesc[100]{Networks~Naming and addressing}
\ccsdesc[100]{Networks~Location based services}

% We no longer use \terms command
%\terms{Theory}

\keywords{DNS, CDN, replica server, subnet}

\maketitle

\section{Introduction} \label{sect:intro}

bla bla

\clearpage

\bibliographystyle{ACM-Reference-Format}
\bibliography{skylines} 

\end{document}
