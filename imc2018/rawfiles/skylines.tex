\documentclass[10pt, sigconf]{acmart}

\let\oldbibitem\bibitem
\def\bibitem{\vfill\oldbibitem}

\usepackage{booktabs} % For formal tables
\usepackage{epsfig,endnotes,subcaption,url}
\usepackage{flushend}


% Copyright
%\setcopyright{none}
%\setcopyright{acmcopyright}
%\setcopyright{acmlicensed}
%\setcopyright{rightsretained}
%\setcopyright{usgov}
%\setcopyright{usgovmixed}
%\setcopyright{cagov}
%\setcopyright{cagovmixed}


\begin{document}
\title[Skylines]{Skylines: Pragmatic, Perspective-Based Internet Mapping}
%\titlenote{Produces the permission block, and copyright information}
%\subtitle{Extended Abstract}

\author{Marc Anthony Warrior}
\affiliation{%
  \institution{Northwestern University}
}
\email{warrior@u.northwestern.edu}

\author{Romain Fontugne}
\affiliation{%
  \institution{IIJ Innovation Institute}
}
\email{romain@iij.ad.jp}
\author{Randy Bush}
\affiliation{%
  \institution{IIJ Innovation Institute}
}
\email{randy@psg.com}

\renewcommand{\shortauthors}{M. Warrior et al.}

\begin{abstract}

    Existing mapping systems and measures used to describe Internet location
    fail to account for complexities introduced by DNS-based content
    allocation schemes. With DNS redirection, in particular, clients residing in
    the exact same network location may potentially be served by completely
    different content sources. Likewise, distant clients may be directed to the
    same servers. The existence of such scenarios challenges the validy of
    some common assumptions about network behavior: the relationship between a
    client's network location and its ``view'' of the Internet, independent of
    performance, is neither predictable nor consistent across clients. 

    In this paper, we introduce skylines, a distance measure that describes
    variation in Internet resource allocation between clients. TODO: finish list
    of contributions

\end{abstract}

%
% The code below should be generated by the tool at
% http://dl.acm.org/ccs.cfm
% Please copy and paste the code instead of the example below. 
%

\copyrightyear{2018} 
\acmYear{2018} 
\setcopyright{acmcopyright}
\acmConference[IMIC '18]{IMC '18: ACM Internet Measurement Conference 2018,
October 31--November 2, 2018}
\acmBooktitle{IMC '18: ACM Internet Measurement Conference 2018,
October 31--November 2, 2018}
\acmPrice{15.00}
\acmDOI{0.0/0.0}
\acmISBN{0-0-0-0-0/0/0}

\begin{CCSXML}
<ccs2012>
    <concept>
        <concept_id>#.#.#</concept_id>
        <concept_desc>Networks~Application layer protocols</concept_desc>
        <concept_significance>300</concept_significance>
    </concept>
    <concept>
        <concept_id>#.#.#</concept_id>
        <concept_desc>Networks~Naming and addressing</concept_desc>
        <concept_significance>100</concept_significance>
    </concept>
    <concept>
        <concept_id>#.#.#</concept_id>
        <concept_desc>Networks~Location based services</concept_desc>
        <concept_significance>100</concept_significance>
    </concept>
</ccs2012>
\end{CCSXML}

\ccsdesc[300]{Networks~Application layer protocols}
\ccsdesc[100]{Networks~Naming and addressing}
\ccsdesc[100]{Networks~Location based services}

% We no longer use \terms command
%\terms{Theory}

\keywords{DNS, CDN, replica server, subnet}

\maketitle

\section{Background \& Motivation} \label{sect:background}
\begin{enumerate}
    \item site performance is dependent on many underlying content sources
        \begin{enumerate}
            \item for Alexa top 500: 50+ domain resolutions per site in median case, 100+ in 80th
                percentile \cite{dnssly}
            \item providers are concerned with the ``bottlenecks'' induced by
                poorly optimized external services (TODO: citations)
            \item providers are interested in coalescing connections where
                possible (TODO: citations; spdy, quic)
        \end{enumerate}
    \item existing measurement and optization efforts tend to focus on
        individual services, often neglecting the complexity of page load times
        and likelihood of external bottlenecks
        \begin{enumerate}
            \item (CDN) resource allocation and mapping algorithm / analysis (TODO: cite
                examples)
            \item client location algorithms and analysis (TODO: cite examples)
        \end{enumerate}
\end{enumerate}

\section{Overview / Goal}
Given that no provider is an island (see \ref{sect:background}), we need to know
how client to sever mapping schemes work \emph{together}, across providers.

Potential applications and insights:
\begin{enumerate}
    \item services may be able to make better decisions about how to allocate
        and/or share infrastructure with other providers
    \item services can better identify potential external sources of outlier behavior
    \item researchers can make more informed decisions in large scale internet
        experiment design and analysis (particularly regarding client grouping)
\end{enumerate}

\section{Approach}
\begin{enumerate}
    \item we DNS resolution to determine mapping schemes, as DNS redirection is
        both widely used and publicly exposed
        \begin{enumerate}
            \item we first identify sites that use DNS redirection 
            \item we then determine the granularity with which sites use DNS
                redirection.
                \begin{enumerate}
                    \item if a site has few unique IPs, it may be using anycast,
                        in which case the details of their mapping scheme is
                        potentially unexposed to the public 
                    \item alternatively, a service with few IPs might actually
                        have few resource locations. There
                        is not much to analyze regarding such a service in the
                        context of this paper, as such a small distribution
                        would inherently have very little overlap with any
                        large scale mapping schemes.
                \end{enumerate}
            \item next, we extract general properties and patterns from the
                remaining, sufficiently granular dataset. We'll use these
                findings to form our distance measure, which will drive the rest
                of our analysis
        \end{enumerate}
    \item we form a distance measure, describing the degree of similarity
        between two clients in terms of the set of web resources they are
        exposed to (in the context of DNS redirection)
    \item ...
\end{enumerate}

\clearpage

\bibliographystyle{ACM-Reference-Format}
\bibliography{skylines} 

\end{document}
